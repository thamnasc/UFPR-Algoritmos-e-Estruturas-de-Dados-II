\documentclass[conference]{IEEEtran}
\usepackage[portuguese]{babel} % vai trocar automaticamente, por exemplo, Table por tabela
\usepackage{cite}
\usepackage{amsmath,amssymb,amsfonts}
\usepackage{algorithmic}
\usepackage{graphicx}
\usepackage{textcomp}
\usepackage{xcolor}
\usepackage{pgfplots}			%Importação de imagens tikz, carregar depois de xcolor
\usetikzlibrary{plotmarks}
%\usetikzlibrary{external}
%\tikzexternalize[prefix=tikz/]
\pgfplotsset{compat=newest}
\pgfkeys{/pgf/number format/.cd,1000 sep={}}%para não colocar , nos tikz para separar milhar

\usepackage[nolist]{acronym}
\usepackage{multirow}

\def\BibTeX{{\rm B\kern-.05em{\sc i\kern-.025em b}\kern-.08em
    T\kern-.1667em\lower.7ex\hbox{E}\kern-.125emX}}
\begin{document}

\begin{acronym}[TDMA]
    \acro{IEEE}{\textit{Institute of Electrical and Electronics Engineers}}
    \acro{ABNT}{Associação Brasileira de Normas Técnicas}
\end{acronym}

\newlength\figureheight
\newlength\figurewidth


\title{Exemplo de Relatório}

\author{\IEEEauthorblockN{Paulo R. Lisboa de Almeida}
\IEEEauthorblockA{\textit{Departamento de Informática} \\
\textit{Universidade Federal do Paraná -- UFPR}\\
Curitiba, Brasil \\
email@servidor.com}
}

\maketitle

\begin{abstract}
O abstract é o resumo do trabalho, onde em poucas linhas a ideia do trabalho é descrita, juntamente com os resultados obtidos.
\end{abstract}

\begin{IEEEkeywords}
exemplo, relatório, estilo, boa escrita
\end{IEEEkeywords}

\section{Introdução}

Esse é um exemplo de relatório, criado utilizando \LaTeX, que é uma ferramenta para criação de textos profissionais e científicos. Outras ferramentas também podem ser utilizadas, como o Microsoft Word e o LibreOffice Writer.

Nesse exemplo, foi utilizado o padrão para conferências do \ac{IEEE} \cite{ieee2022}. É possível a utilização de qualquer outro padrão popular, como o da \ac{ABNT} \cite{abnt2022}. Existem modelos no padrão \ac{ABNT} para \LaTeX\ disponíveis em \cite{abntex2022}. É possível também encontrar modelos para o padrão \ac{ABNT} (e outros) para Microsoft Word e Libreoffice Writer facilmente na internet.

\section{Seções}

O título e o nome do autor sempre devem existir. Os demais itens do cabeçalho, como departamento e organização, são opcionais em relatórios simples. O abstract e palavras chave (chamadas de \textit{Index Terms} no padrão \ac{IEEE}), também são opcionais em um relatório simples.

As demais seções dependem do texto a ser produzido, mas geralmente os documentos possuem pelo menos uma seção de introdução, e uma de conclusão. Outros exemplos de seção são seções para se definir o protocolo experimental, experimentos realizados, discussão de resultados, \dots

\section{Boas Práticas}

``Se eu tivesse mais tempo, te escreveria uma carta mais curta'' (Blaise Pascal). Seja conciso, passando as ideias de maneira direta (ou no bom português, sem ``encher linguiça''). Mas tome cuidado para não ocultar nenhum detalhe relevante.

Textos técnicos e científicos devem ser escritos de forma indireta. Não utilize, por exemplo, narrativa em primeira pessoa (e.g. ``Eu fiz os experimentos \dots'' deve ser substituído por ``Os experimentos foram realizados \dots'').

Sempre que possível use referências bibliográficas confiáveis (e.g. livros ou artigos científicos) para suportar suas afirmações. Veja um exemplo a seguir:

Em se tratando se sistemas operacionais, os sistemas baseados em UNIX foram capazes de demonstrar já na década de 70 que um sistema operacional interativo e poderoso não precisa ser custoso em termos de hardware e recursos humanos \cite{ritchieThompson1978}.

Note que o formato da citação depende do padrão sendo utilizado. No padrão \ac{IEEE} usa-se [NUMERO]. Já o padrão \ac{ABNT} utiliza o formato (AUTOR, ano).


\subsection{Figuras e Tabelas}

A Tabela \ref{table:amplaConc} é um exemplo de tabela. Não deixe tabelas ou figuras ``soltas'' no texto. Tabelas e figuras devem possuir uma legenda que indique de forma clara do que se tratam os dados.

Sempre cite e comente as figuras e tabelas no texto. Por exemplo, na Tabela \ref{table:amplaConc} é possível verificar uma queda na relação candidato/vaga com o decorrer dos anos.


\begin{table}[htpb]
\centering
\caption{Relação Candidato/Vaga em ampla concorrência.}
\label{table:amplaConc}
\begin{tabular}{lllll}
 & \multicolumn{4}{c}{Candidato/Vaga por ano} \\\cline{2-5}
Curso                       & 2016      & 2017     & 2019     & 2021     \\\hline
BCC                    & 14,2      & 14,2     & 12,2     & 9,8      \\
Eng. Civil             & 15,0      & 13,1     & 6,2      & 7,7\\\hline
\end{tabular}
\end{table}

Muitas vezes os dados podem ser melhor expressados utilizando figuras. A Figura \ref{fig:amplaConc} monstra a mesma informação da Tabela \ref{table:amplaConc}, mas em formato de gráfico. A Figura foi feita utilizando Tikz para \LaTeX. O Tikz é uma ferramenta profissional para criar gráficos vetoriais em \LaTeX (faça um \textit{zoom} nesse gráfico e note que ele não perde qualidade). É possível (e mais fácil) importar figuras jpeg ou png diretamente para o relatório.

\begin{figure}[htpb]
        \centering
        \setlength\figureheight{3.0cm}
		\setlength\figurewidth{8.0cm}
        % This file was created by matlab2tikz.
% Minimal pgfplots version: 1.3
%
%The latest updates can be retrieved from
%  http://www.mathworks.com/matlabcentral/fileexchange/22022-matlab2tikz
%where you can also make suggestions and rate matlab2tikz.
%
\begin{tikzpicture}

\begin{axis}[%
width=0.95092\figurewidth,
height=\figureheight,
at={(0\figurewidth,0\figureheight)},
scale only axis,
xlabel={Ano},
ylabel={Candidato/Vaga},
%legend style={at={(0.97,0.03)},anchor=south east,legend cell align=left,align=left,draw=white!15!black},
y label style={at={(axis description cs:-0.1,.5)},anchor=south},
]

\addplot [color=blue,solid,mark=+, mark options={solid}]
table[row sep=crcr]{%
	2016	14.2\\
	2017	14.2\\
	2019	12.2\\
	2021	9.8\\
};
\addlegendentry{BCC};

\addplot [color=red,densely dotted]
table[row sep=crcr]{%
	2016	15.0\\
	2017	13.1\\
	2019	6.2\\
	2021	7.7\\
};
\addlegendentry{Eng. Civil};

\end{axis}
\end{tikzpicture}%
        \caption{Gráfico com a relação Candidato/Vaga em ampla concorrência.}
        \label{fig:amplaConc}
\end{figure}

Uma boa análise (de exemplo) para a Figura \ref{fig:amplaConc} pode ser: na Figura \ref{fig:amplaConc} pode-se observar o declínio da relação candidato/vaga nos últimos anos, sendo que esse declínio é mais severo para o curso de Engenharia Civil, quando comparado com o curso de Ciência da Computação.

\section{Glossário}

A abreviação e.g. vem do latim \textit{exempli gratia}, e significa \textit{por exemplo} em português. É usual utilizar \textit{p. ex.} no lugar de e.g. em português.

A abreviação i.e. vem do latim \textit{id est}, que significa \textit{isto é} ou \textit{ou seja} em português.

Ambas expressões são comuns em textos científicos e jurídicos.

\section{Conclusão}

Este é apenas um exemplo de relatório para os ``não iniciados'' em textos técnicos e científicos. O padrão utilizado (\ac{ABNT}, \ac{IEEE}, \dots) depende da especificação do trabalho ou projeto. Para trabalhos genéricos, onde não há especificação de padrão, basta utilizar um padrão amplamente aceito, como o utilizado neste texto.

A ferramenta a ser utilizada para a confecção do texto, como \LaTeX, Microsoft Word ou Libreoffice, fica a cargo da preferência do escritor. No entanto, opte por gerar o texto final em formato pdf para garantir a compatibilidade.

E sempre lembre-se de que o mais importante em um trabalho qualquer é o conteúdo. Seja direto, e passe suas ideias da forma mais clara o possível.

\bibliographystyle{IEEEtran}
\bibliography{bibliography}

\end{document}
